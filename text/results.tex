Totally, we found 17 peaks and their corresponding energy, listed as the table below:

\begin{table}[h!]
  \begin{center}
    \caption{Peaks with Corresponding Energy.}
    \label{tab:table1}
    \begin{tabular}{l|l|l}
      \textbf{Peak \#} & \textbf{Channel \#} & \textbf{Energy(keV)}\\
      \hline
      1 & 207.72934164066072 & 59.5409\\
      2 & 2353.969595281511 & 661.657\\
      3 & 4177.6835240821465 & 1173.228\\
      4 & 4745.511220436352  & 1332.492\\
      5 & 429.56728236065885 & 121.7817\\
      6 & 867.77043054316  & 244.6974\\
      7 & 1222.6117121691966  & 344.2785\\
      8 & 1460.7650922753362  & 411.1165\\
      9 & 2772.003206539464  & 778.9045\\
      10 & 3432.077389155436 & 964.072\\
      11 & 5014.927456550307 & 1408.013\\
      12 & 3959.8315489345346 & 1112.076\\
      13 & 284.28923089774975 & 80.9979\\
      14 & 980.9088851006587  & 276.3989\\
      15 & 1075.1587761476976 & 302.8508\\
      16 & 1264.6138573728588 & 356.0129\\
      17 & 1363.8190002015974 & 383.8485\\
 
    \end{tabular}
  \end{center}
\end{table}
Making a linear regression of the table above, we have
\[Energy[keV]=0.280521278253 \cdot Channel\,Number+1.27621057811\,\,\,\,\,\,\,\,\,(1)\]
with the coefficient of determination ${R^2}$:
\[{R^2}=0.999999997516\]
which is very close to 1. so the calibration result is very good.


